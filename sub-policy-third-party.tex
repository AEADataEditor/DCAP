\section{Policy and Protocol on Third-Party Verifications}\label{policy-and-protocol-on-third-party-verifications}

This protocol%
\footnote{Also available on the web at \url{https://www.aeaweb.org/journals/data/policy-third-party}.}
describes how third parties can, at the request of the AEA
Data Editor, conduct a reproducibility check. 
%
Alternate protocols are possible, but should be verified with the AEA
Data Editor prior to engaging any resources.

\subsection{Preliminaries}\label{preliminaries}

\begin{itemize}
\tightlist
\item
  The author(s) should provide a complete and exhaustive archive, ready
  for publication, to the AEA Data Editor.

  \begin{itemize}
  \tightlist
  \item
    The archive does not need to be public at this stage, as long as it
    can be shared privately.
  \item
    The archive does not have to contain the data necessary for the
    reproducibility check if data is confidential or proprietary.
    However, the archive must contain a publishable description of how
    an independent researcher can access the data (``data availability
    statement'') once all conditions for access have been met.
  \end{itemize}
\item
  The AEA Data Editor will verify the archive.

  \begin{itemize}
  \tightlist
  \item
    Aside from the data itself, all other materials specified in the
    \hyperref[data-and-code-availability-policy]{AEA Data and Code Availability Policy}, such as data
    citations, must be included.
  \end{itemize}
\end{itemize}

\subsection{What or Who Is a Third-Party
Replicator}\label{what-or-who-is-a-third-party-replicator}

The third-party replicator is a person not affiliated with the AEA
editorial offices who has access to the (public-use, confidential or
restricted) data used in the author's paper, and who

\begin{itemize}
\tightlist
\item
  has not been involved with the author's research project
\item
  has disclosed any conflicts of interest
\item
  promises to conduct an arms-length reproducibility exercise with no
  direct interaction with the author (other than to undertake specific
  steps to access the data)
\end{itemize}

\subsection{Steps for the Third-Party Replicator}\label{steps-for-the-third-party-replicator}

\begin{itemize}
\tightlist
\item
  Download the author's replication archive(s) from the location
  provided by the AEA Data Editor (public or privately shared).

  \begin{itemize}
  \tightlist
  \item
    Do not obtain any code or instructions directly from the author.
  \end{itemize}
\item
  Ensure access to any confidential files that are described in the
  replication archive's README.

  \begin{itemize}
  \tightlist
  \item
    Consider whether a person not familiar with the precise data
    environment could reasonably find and access the data, based solely
    on the instructions in the README.
  \end{itemize}
\item
  Follow the
  \urlcite{https://social-science-data-editors.github.io/guidance/Verification_guidance.html}{checklist}
  to conduct the reproducibility exercise, relying exclusively on the
  README for instructions and guidance.
\item
  Write a
  \purlcite{https://github.com/AEADataEditor/replication-template/blob/master/REPLICATION.md}{report}
\item
  Send the report to the AEA Data Editor.
\item
  Report any interactions with the author in the course of conducting
  the reproducibility exercise (help, assistance, clarifications).
\end{itemize}

~
