\section{Data and Code Availability Policy}\label{data-and-code-availability-policy}

\textbf{It is the policy of the American Economic Association to publish
papers only if the data and code used in the analysis are clearly and
precisely documented and access to the data and code is non-exclusive to
the authors.}

\textbf{Authors of accepted papers that contain empirical work,
simulations, or experimental work must provide, prior to acceptance,
information about the data, programs, and other details of the
computations sufficient to permit replication, as well as information
about access to data and programs.}

Data and programs should be archived in the
\emph{AEA Data and Code
Repository}.%
\footnote{\url{https://www.openicpsr.org/openicpsr/aea}. Other repositories and archives may be
	acceptable, as long as these are considered to be ``trusted'' archives or
	repositories, see
	\furlcite{https://social-science-data-editors.github.io/guidance/Requested_information_hosting.html}{guidance}.
	The AEA Data Editor will assess suitability of any such repositories and
	archives.} 
Authors will provide access to
editors and reviewers, if requested, to both data and programs prior to
acceptance. The Editor should be notified at the time of submission if
access to the data used in a paper is restricted or limited, or if, for
some other reason, the requirements above cannot be met.

If data or programs cannot be published in an openly accessible trusted
data repository, authors must commit to preserving data and code for a
period of no less than five years following publication of the
manuscript, and to providing reasonable assistance to requests for
clarification and replication.

The AEA Data Editor will assess compliance with this policy, and will
verify the accuracy of the information prior to acceptance by the
Editor.

\subsection{Content and Scope}\label{content-and-scope}

For econometric, simulation, and experimental papers, the replication
materials shall include (a) the data set(s), (b) description sufficient
to access all data at their original source location, (c) the programs
used to create any final and analysis data sets from raw data, (d)
programs used to run the final models, and (e) description sufficient to
allow all programs to be run.

For papers collecting original data through surveys or experiments, the
replication materials shall also include (f) survey instruments or
experiment instructions, (g) computer code for experiment or survey
collection mechanisms, and (h) original instructions and details on
subject selection. See the supplementary 
\hyperref[policy-for-papers-conducting-experiments-and-collecting-primary-data]{Policy for Papers Conducting Experiments and Collecting Primary  Data}.

\subsubsection{Data and Software
Citations}\label{data-and-software-citations}

All source data used in the paper shall be cited, following the
\purlcite{https://www.aeaweb.org/journals/policies/sample-references}{AEA Sample References}
Citation of software packages is also encouraged.

\subsubsection{Data Availability
Statement}\label{data-availability-statement}

A data availability statement covering both the source data and any
derivative data shall be provided in the README file. It may
additionally be provided as part of online appendices. The data
availability statement shall provide detailed information on how, where,
and under what conditions an independent researcher can access the
original source data, as well as author-generated derivative data, and
must be explicit and accurate about any restrictions, requirements,
payments, and processing delays. The data availability statement shall
provide information to assure the reader that the data are available for
a sufficiently long period of time.

\subsubsection{Non-Public Data}\label{non-public-data}

This policy, with the exception of item (a) above, also applies to
papers that use data that cannot be published as part of a replication
package or in an openly accessible trusted data repository. Examples
include confidential data with identifying information of persons or
businesses and data subject to data use agreements or copyrights that
prohibit redistribution. When possible, a private (not to be published)
version of the data should be provided to the AEA Data Editor and/or a
designated third-party replicator who can provide a third-party reproducibility report (see \hyperref[policy-and-protocol-on-third-party-verifications]{Policy and Protocol on Third-Party Verifications}).

\subsubsection{Formats}\label{formats}

\textbf{Data:} The data files may be provided in any format compatible
with any commonly used statistical package or software. Authors are
encouraged to provide data files in open, non-proprietary formats.
Authors should ensure that a meaningful name or description (label) is
available for every variable in the provided datasets. Codebooks or
similar metadata should describe the allowed values and their meaning
for each variable. It is acceptable to reference publicly available
documentation for these items.

\textbf{Code:} The programs may be provided in any format compatible
with commonly used statistical package or software. Should unusual or
costly software be required, authors are required to notify the AEA Data
Editor. A master script is strongly encouraged.

\subsubsection{Metadata}\label{metadata}

As part of the archive, authors must provide a README file listing all
included files and documenting the purpose, format, and provenance of
each file provided, as well as instructing a user on how replication can
be conducted. The README shall contain the data availability statement
and proper citations for all data used.

The README shall follow the schema provided by the
\purlcite{https://social-science-data-editors.github.io/guidance/template-README.html}{Social
Science Data Editors' template README}

Common formats are txt, PDF, and Markdown. The README file should not
require proprietary software to view.

\subsubsection{Version of Record}\label{version-of-record}

After the data and code deposit is accepted by the AEA Data Editor, it
will become the version of record associated with the paper. Corrections
and revisions are subject to the \hyperref[policy-on-revisions-of-data-and-code-deposits-in-the-aea-data-and-code-repository]{Policy on Revisions of Data and Code Deposits in the AEA Data and Code Repository}.

\subsection{Registration of Randomized Control
Trials}\label{registration-of-randomized-control-trials}

It is the policy of the AEA that randomized control trials must be
registered on the RCT Registry. All such registrations shall be cited in
the title footnote and elsewhere in the paper as appropriate. Please see
the \purlcite{https://www.aeaweb.org/journals/policies/rct-registry}{RCT
Registry policy}

\subsection{Ethics Approval}\label{ethics-approval}

If applicable, approval by ethics boards---the Institutional Review
Board (IRB) in the United States and equivalent institutions
elsewhere---should be demonstrated by including the name of the ethics
board and any approval or exemption record number in the title footnote
and the author disclosure statement(s). See the
\purlcite{https://www.aeaweb.org/journals/policies/disclosure-policy}{Disclosure
Policy}

\subsection{Instructions}\label{instructions}

Detailed instructions for preparing and depositing replication packages
are provided in the
\purlcite{https://aeadataeditor.github.io/aea-de-guidance/step-by-step.html}{AEA
Data Editor's step-by-step guide}

For more information, see
\furlcite{https://www.aeaweb.org/journals/data/faq}{Frequently Asked
Questions}.


\emph{This version (September 2020) supplants all
\href{https://www.aeaweb.org/journals/data/archive}{prior data
policies}.}

~
