% PP-Article.tex for AEA last revised 22 June 2011
\documentclass[PP]{AEA}

%%%%%% NOTE FROM OVERLEAF: The mathtime package is no longer publicly available nor distributed. We recommend using a different font package e.g. mathptmx if you'd like to use a Times font.
\usepackage{mathptmx}
\usepackage{url}
\usepackage{hyperref}

% If you have trouble with the mathtime package please see our technical support 
% document at: http://www.aeaweb.org/templates/technical_support.pdf
% You may remove the mathtime package if you can't get it working but your page
% count may be inaccurate as a result.
% \usepackage[cmbold]{mathtime}

% Note: you may use either harvard or natbib (but not both) to provide a wider
% variety of citation commands than latex supports natively. See below.

% Uncomment the next line to use the natbib package with bibtex 
%\usepackage{natbib}

% Uncomment the next line to use the harvard package with bibtex
%\usepackage[abbr]{harvard}

% This command determines the leading (vertical space between lines) in draft mode
% with 1.5 corresponding to "double" spacing.
\draftSpacing{1.5}

%\newcommand{\Href}[2]{\href{#1}{#2}}
% alternate version:
\newcommand{\Href}[2]{\href{#1}{#2} (\url{#1})}
% alternate version 2:
%\newcommand{\Href}[2]{\href{#1}{#2}\footnote{\url{#1}}}



\begin{document}

\title{Data and Code Availability Policy}
\shortTitle{Data and Code Avail. Policy}
\author{American Economic Association\thanks{AEA Address. For questions, contact the AEA Data Editor at dataeditor@aeapubs.org}}\pubMonth{Month}
\date{\today}
\pubYear{Year}
\pubVolume{Vol}
\pubIssue{Issue}
\maketitle


It is the policy of the American Economic Association to publish papers only if the data and code used in the analysis are clearly and precisely documented, and access to the data and code is clearly and precisely documented and is non-exclusive to the authors.

Authors of accepted papers that contain empirical work, simulations, or experimental work must provide, prior to acceptance, information about the data, programs, and other details of the computations sufficient to permit replication, as well as information about access to data and programs.

Data and programs should be archived in the \textit{AEA Data and Code Repository}. Authors will provide access to editors and reviewers, if requested, to both data and programs prior to acceptance. The Editor should be notified at the time of submission if access to the data used in a paper is restricted or limited, or if, for some other reason, the requirements above cannot be met. The AEA Data Editor will assess compliance with this policy, and will verify the accuracy of the information prior to acceptance by the Editor.

\section{Instructions}

When requested by the Editor during the refereeing process, authors are expected to provide location and access details for their data, programs, and  replication instructions. Final files may be deposited in the \Href{https://www.openicpsr.org/openicpsr/aea}{\textit{AEA Data and Code Repository}}.\footnote{Other repositories and archives may be acceptable, as long as these are considered to be "trusted" archives or repositories, see \Href{https://social-science-data-editors.github.io/guidance/Requested_information_hosting.html}{guidance}. The AEA Data Editor will assess suitability of any such repositories and archives.}

The AEA Data Editor will verify all information prior to acceptance of the manuscript by the Editor.

\section{Content and Scope}

For econometric, simulation, and experimental papers, the replication materials shall include (a) the data set(s), (b) the programs used to create any final and analysis data sets from raw data, (c) programs used to run the final models, and (d) description sufficient to allow all programs to be run.

\section{Data}

For data, enough information should be provided (a) to accurately describe the data so that somebody who doesn’t have knowledge of the data can understand its principal (and salient)  characteristics (INFORMATION); (b) to be able to acquire the data (whether by download, by contract, by application process, etc.) (ACCESSIBILITY); and (c) to assure the reader that the data is available for a sufficiently long period of time (PERSISTENCE). 

The data files can be provided in any format compatible with any commonly used statistical package or software. Authors are encouraged to provide data files in open, non-proprietary formats.
Programs

Authors should provide clear documentation of all code (within the code/script files themselves, and through a README). In particular: (a) it should be clear from the code (and/or the README) where to find the information contained in each table, figure, and in-text number; (b) all pre-requisites (data, code, programs, software, possibly operating system) should be identified (including version numbers); and (c) where appropriate, random seeds should be fixed.

The programs can be provided in any format compatible with commonly used statistical package or software. Should unusual or costly software be required, the Authors are required to notify the AEA Data Editor.

A master script is strongly encouraged.

\section{Metadata}

As part of the archive, authors must provide a README file listing all included files and documenting the purpose, format, and provenance of each file provided, as well as instructing a user on how replication can be conducted.

Common formats are txt, PDF, and Markdown. The README file should not require proprietary software to view. It should guide a user on the types of files and how to use them to do replication.

Authors should ensure that a meaningful name or description (label) is available for every variable in provided datasets. Codebooks or similar metadata should describe the allowed values and their meaning for each variable. It is acceptable to reference publicly available documentation for these items.

\section{Procedure for restricted-access data}

If data is subject to any access restriction that prevents authors from depositing the files in an openly accessible data repository, additional information is required. Authors shall provide detailed information on how, where, and under what conditions an independent researcher can access the data. This information shall be provided to the editors upon submission, and shall be part of the README. All programs, as described above, shall still be provided.

The AEA Data Editor will verify the information, and may contact the data providers identified by the authors.

\section{Special rules for experimental papers}

For experimental papers, additional rules apply. We normally expect authors of experimental articles to supply the following  supplementary materials (any exceptions to this policy should be requested at the time of submission):

\subsection{Original Instructions}

The original instructions should be summarized as part of the discussion of experimental design in the submitted manuscript, and also provided in full as an appendix at the time of submission. The instructions should be presented in a way that, together with the design summary, conveys the protocol clearly enough that the design could be replicated by a reasonably skilled experimentalist. For example, if different instructions were used for different sessions, the correspondence should be indicated.

\subsection{Subject Selection}

Information about subject eligibility or selection,  such as exclusions based on past participation in experiments, college major, etc. This should be summarized as part of the discussion of experimental design in the submitted manuscript.

\subsection{Software and Scripts}

Any computer programs, configuration files, or scripts used  to run the experiment shall be provided. These should be summarized as appropriate in the submitted manuscript and deposited in the AEA Data and Code Repository. All requirements noted above for programs apply.

\subsection{Raw Data}

The raw data from the experiment should be summarized as appropriate in the submitted manuscript, and deposited in the AEA Data and Code Repository. 

We strongly encourage authors to deposit raw data and instructions separately from other replication materials, in order to provide greater visibility to the author’s work.

\subsection{Analysis Programs}

All requirements for final and intermediate data files, as well as cleaning and analysis programs, outlined above, continue to apply. (See "Data" section above.)

\section{Registration}

The AEA has a policy on the registration of randomized control trials. Please see the policy at \url{https://www.aeaweb.org/journals/policies/rct-registry}.

\section{Transmission of Data and Code Materials}

Instructions on how to deposit materials at the \textit{AEA Data and Code Repository} are provided at \Href{https://www.openicpsr.org/openicpsr/aea/deposit-instructions}{openICPSR}, with additional instructions available at \url{https://aeadataeditor.github.io/aea-de-guidance/data-deposit-aea-guidance.html}.

Files uploaded to the \textit{AEA Data and Code Repository} should retain the file names as originally executed or used, their original file format, and their original "grouping" in terms of directories.

The \textit{AEA Data and Code Repository} can handle files up to 2GB. Please \Href{https://www.openicpsr.org/openicpsr/contactUs}{contact the repository staff} should you need to upload larger files, or encounter any problems.


For more information, see the \Href{https://www.aeaweb.org/journals/policies/data-code/faq}{Frequently Asked Questions}.

\appendix 
\section{Note}
This  \textit{AEA Data and Code Availability Policy} was published online at \url{https://www.aeaweb.org/journals/policies/data-code} on July 16, 2019,\footnote{See \url{https://www.aeaweb.org/news/member-announcements-july-16-2019}.} replacing previous policies. The policy is reprinted here as the version of record. Guidance to authors on how to comply with the policy can be found at the web address above, and at \url{https://aeadataeditor.github.io/aea-de-guidance/data-deposit-aea-guidance.html}. 

% Remove or comment out the next two lines if you are not using bibtex.
%\bibliographystyle{aea}
%\bibliography{BibFile}

% The appendix command is issued once, prior to all appendices, if any.
%\appendix

%\section{Mathematical Appendix}

\end{document}

